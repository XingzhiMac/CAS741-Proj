\documentclass{article}

\usepackage{tabularx}
\usepackage{booktabs}

\title{CAS 741: Problem Statement\\Radio Signal Strength Calculator}

\author{Xingzhi Liu\\liux57}

\date{September 21, 2020}

%% Comments

\usepackage{color}

\newif\ifcomments\commentstrue

\ifcomments
\newcommand{\authornote}[3]{\textcolor{#1}{[#3 ---#2]}}
\newcommand{\todo}[1]{\textcolor{red}{[TODO: #1]}}
\else
\newcommand{\authornote}[3]{}
\newcommand{\todo}[1]{}
\fi

\newcommand{\wss}[1]{\authornote{blue}{SS}{#1}} 
\newcommand{\wxz}[1]{\authornote{cyan}{XZ}{#1}} 
\newcommand{\plt}[1]{\authornote{magenta}{TPLT}{#1}} %For explanation of the template
\newcommand{\an}[1]{\authornote{cyan}{Author}{#1}}


\begin{document}

\maketitle

\begin{table}[hp]
\caption{Revision History} \label{TblRevisionHistory}
\begin{tabularx}{\textwidth}{llX}
\toprule
\textbf{Date} & \textbf{Developer(s)} & \textbf{Change}\\
\midrule
Sept. 21, 2020 & Xingzhi Liu & Added problem statement\\
\bottomrule
\end{tabularx}
\end{table}

\section{The problem} This project aims to estimate radio signal strength 
in an indoor environment: given the location, power, and radio frequency 
of a radio transmitter, and given the floor plan of the building where the 
transmitter is installed, our goal is to determine the radio signal strength 
within the building. 

\section{Why is it important?} 
From the evaluation of Wi-Fi signal coverage to indoor localization, radio 
signal strength data is essential in various application scenarios of indoor 
wireless systems. However, collecting signal strength data in the real world 
is both expensive and time-consuming. This project will help us obtain signal 
strength without conducting on-field surveys, thus reducing the time and cost 
of data collection.

\section{Context of the problem}
Wireless system engineers are our software's potential users. The potential 
users are estimated to have basic knowledge of wireless communication and 
software development.

\end{document}
