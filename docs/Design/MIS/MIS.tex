\documentclass[12pt, titlepage]{article}

\usepackage{amsmath, mathtools}

\usepackage[round]{natbib}
\usepackage{amsfonts}
\usepackage{amssymb}
\usepackage{graphicx}
\usepackage{colortbl}
\usepackage{xr}
\usepackage{hyperref}
\usepackage{longtable}
\usepackage{xfrac}
\usepackage{tabularx}
\usepackage{float}
\usepackage{siunitx}
\usepackage{booktabs}
\usepackage{multirow}
\usepackage[section]{placeins}
\usepackage{caption}
\usepackage{fullpage}

\hypersetup{
bookmarks=true,     % show bookmarks bar?
colorlinks=true,       % false: boxed links; true: colored links
linkcolor=red,          % color of internal links (change box color with linkbordercolor)
citecolor=blue,      % color of links to bibliography
filecolor=magenta,  % color of file links
urlcolor=cyan          % color of external links
}

\usepackage{array}

\externaldocument{../../SRS/SRS}

\input{../../Comments}
\input{../../Common}

\begin{document}

\title{Module Interface Specification for RSSC}

\author{Xingzhi Liu}

\date{\today}

\maketitle

\pagenumbering{roman}

\section{Revision History}

\begin{tabularx}{\textwidth}{p{3cm}p{2cm}X}
\toprule {\bf Date} & {\bf Version} & {\bf Notes}\\
\midrule
Nov. 19, 2020 & 1.0 & Initial Release\\
Dec. 18, 2020 & 1.1 & Revision 1\\
\bottomrule
\end{tabularx}

~\newpage

\section{Symbols, Abbreviations and Acronyms}

See SRS Documentation at \url{https://github.com/XingzhiMac/CAS741-Proj/}\\

The table that follows summarizes the symbols used in this document that are not
mentioned in SRS.

\renewcommand{\arraystretch}{1.2}
%\noindent \begin{tabularx}{1.0\textwidth}{l l X}
\noindent \begin{longtable*}{l l p{12cm}} \toprule
\textbf{symbol} & \textbf{unit} & \textbf{description}\\
\midrule 
$\forall$ & - & for all
\\
$\neg$ & - & not
\\
$\Rightarrow$ & - & implies/leads to
\\
$:=$ & - & equal by definition
\\
\bottomrule
\end{longtable*}

\newpage

\tableofcontents

\newpage

\pagenumbering{arabic}

\section{Introduction}

The following document details the Module Interface Specifications for
Radio Signal Strength Calculator. It is intended to ease navigation through 
the program for design and maintenance purposes.

Complementary documents include the System Requirement Specifications
and Module Guide.  The full documentation and implementation can be
found at \url{https://github.com/XingzhiMac/CAS741-Proj/}. 

\section{Notation}

The structure of the MIS for modules comes from \citet{HoffmanAndStrooper1995},
with the addition that template modules have been adapted from
\cite{GhezziEtAl2003}.  The mathematical notation comes from Chapter 3 of
\citet{HoffmanAndStrooper1995}.  For instance, the symbol := is used for a
multiple assignment statement and conditional rules follow the form $(c_1
\Rightarrow r_1 | c_2 \Rightarrow r_2 | ... | c_n \Rightarrow r_n )$.

The following table summarizes the primitive data types used by RSSC. 

\begin{center}
\renewcommand{\arraystretch}{1.2}
\noindent 
\begin{tabular}{l l p{7.5cm}} 
\toprule 
\textbf{Data Type} & \textbf{Notation} & \textbf{Description}\\ 
\midrule
Boolean & Boolean & a 1-bit data with two possible values (0 and 1)\\
character & char & a single symbol or digit\\
integer & $\mathbb{Z}$ & a number without a fractional component in (-$\infty$, $\infty$) \\
natural number & $\mathbb{N}$ & a number without a fractional component in [1, $\infty$) \\
non-negative integer & $\mathbb{N}_0$ & a number without a fractional component in [0, $\infty$) \\
real & $\mathbb{R}$ & any number in (-$\infty$, $\infty$)\\

\bottomrule
\end{tabular} 
\end{center}

\noindent
The specification of RSSC uses some derived data types: sets, strings, and
tuples. Sets are lists filled with elements of the same data type. In this document, a set of data in type T is represented as set[T]. Strings are lists of characters. Tuples contain a list of values, potentially of different types.

In addition, RSSC defines the following classes as its unique data types: Point (defined in
\autoref{pointModule}), Wall (defined in \autoref{wallModule}), FloorMap (defined in \autoref{floorMap}), LinearPath (defined in \autoref{linearPathModule}), LineOfSight (defined in \autoref{losModule}), and FirstOrderReflection (defined in \autoref{forsModule}). 

\section{Module Decomposition}

The following table is taken directly from the Module Guide document for this project.

\begin{table}[h!]
\centering
\begin{tabular}{p{0.3\textwidth} p{0.6\textwidth}}
\toprule
\textbf{Level 1} & \textbf{Level 2}\\
\midrule

{Hardware-Hiding} & ~ \\
\midrule

\multirow{3}{0.3\textwidth}{Behaviour-Hiding} & Input Parameters\\
& Output\\
& Control Module\\
& Received Signal Strength\\
& Specification Parameters Module\\
\midrule

\multirow{3}{0.3\textwidth}{Software Decision} & Point\\
& Wall\\
& Floor Map\\
& Equation Finder\\
& Linear Signal Path\\
& Intersection\\
& Linear Path Loss\\
& Line-Of-Sight Signal\\
& Specular Reflection\\
& First Order Reflection Signal\\
\bottomrule

\end{tabular}
\caption{Module Hierarchy}
\label{TblMH}
\end{table}

\newpage
~\newpage

\section{MIS of Control Module} \label{controlModule}

\subsection{Module}

main

\subsection{Uses}

Param (\autoref{inputParamModule}), FloorMap(\autoref{floorMap}), ReceivedSignalStrength(\autoref{rssModule}), Output(\autoref{outputModule})

\subsection{Syntax}

\subsubsection{Exported Constants}

\subsubsection{Exported Access Programs}

\begin{center}
\begin{tabular}{p{2cm} p{4cm} p{4cm} p{2cm}}
\hline
\textbf{Name} & \textbf{In} & \textbf{Out} & \textbf{Exceptions} \\
\hline
main & - & - & - \\
\hline
\end{tabular}
\end{center}

\subsection{Semantics}

\subsubsection{State Variables}

None

\subsubsection{Access Routine Semantics}

\noindent main():
\begin{itemize}
\item transition: Modify the state of Param module and the environment variables for the
Output module by following these steps:\\

	Get (filenameTSM: string), (filenameSP: string), (filenameWALL: string), and (filenameOut: string) from user\\
	
	Param.load\_params(filenameTSM, filenameSP, filenameWALL)\\
	
	$map$ = FloorMap.create()\\
	
	$[P_{sp}^{dBm}]$ := empty set\\
	For ($sampling\_point$: Point) in Param.$[Pos_{sp}]$:\\
	$P_{sp}^{dBm}$ := ReceivedSignalStrength.get\_received\_power($map$, $sampling\_point$)\\
	Add $P_{sp}^{dBm}$ into the set $[P_{sp}^{dBm}]$\\
	
	Output.output(filenameOut, Param.$[Pos_{sp}]$, $[P_{sp}^{dBm}]$)
\end{itemize}

\newpage


\section{MIS of Input Parameters Module} \label{inputParamModule}

\subsection{Module}

Param

\subsection{Uses}

SpecParam (\autoref{specParamModule}), Point (\autoref{pointModule})

\subsection{Syntax}

\subsubsection{Exported Constants}

None

\subsubsection{Exported Access Programs}

\begin{center}
\begin{tabular}{p{4cm} p{2cm} p{2cm} p{4cm}}
\hline
\textbf{Name} & \textbf{In} & \textbf{Out} & \textbf{Exceptions} \\
\hline
load\_params & s1: string, s2: string, s3: string & - & FileError \\
verify\_params & - & - & badTransmittance, badReflectance, badTransmissionPower, badSignalFrequency, badPosition, inconsistentWallParams \\
$Pos_{tsm}$ & - & Point & - \\
$[Pos_{sp}]$ & - & set[Point] & - \\
$[C]$ & - & set[Point] & - \\
$[D]$ & - & set[Point] & - \\
$[T]$ & - & set[$\mathbb{R}$] & - \\
$[R]$ & - & set[$\mathbb{R}$] & - \\
$P_{tsm}^{dBm}$ & - & $\mathbb{R}$ & - \\
$f$ & - & $\mathbb{R}$ & - \\
\hline
\end{tabular}
\end{center}

\subsection{Semantics}

\subsubsection{State Variables}
$Pos_{tsm}$: Point\\
$[Pos_{sp}]$: set[Point]\\
$[C]$: set[Point]\\
$[D]$: set[Point]\\
$[T]$: set[$\mathbb{R}$]\\
$[R]$: set[$\mathbb{R}$]\\
$P_{tsm}^{dBm}$: $\mathbb{R}$\\
$f$: $\mathbb{R}$\\
$length_C$: $\mathbb{R}$\\
$length_D$: $\mathbb{R}$\\
$length_T$: $\mathbb{R}$\\
$length_R$: $\mathbb{R}$\\
\subsubsection{Environment Variables}
tsmFile: set[string]\\
spFile: set[string]\\
wallFile: set[string]\\
\subsubsection{Assumptions}
\begin{itemize}
\item load\_params will be called before any of the state variables be accessed.
\item tsmFile contains the string equivalents of the numeric values for $Pos_{tsm}$, $P_{tsm}^{dBm}$ and $f$, each on a new line.
\item spFile contains the string equivalents of elements in the user-input item $[Pos_{sp}]$ , each on a new line, in the form of two numbers separated with a comma.
\item wallFile contains the string equivalents of elements in the user-input items $[C]$, $[D]$, $[T]$, and $[R]$. Each line is in the form of 6 numbers separated by 5 commas (each line should be “$x_{C_x}, y_{C_x}, x_{D_x}, y_{D_x}, T_x, R_x$”).

\end{itemize}

\subsubsection{Access Routine Semantics}

\noindent Param.$Pos_{tsm}$():
\begin{itemize}
\item output: $out := Pos_{tsm}$
\item exception: none
\end{itemize}

\noindent Param.$[Pos_{sp}]$():
\begin{itemize}
\item output: $out := [Pos_{sp}]$
\item exception: none
\end{itemize}

\noindent Param.$[C]$():
\begin{itemize}
\item output: $out := [C]$
\item exception: none
\end{itemize}

\noindent Param.$[D]$():
\begin{itemize}
\item output: $out := [D]$
\item exception: none
\end{itemize}

\noindent Param.$[T]$():
\begin{itemize}
\item output: $out := [T]$
\item exception: none
\end{itemize}

\noindent Param.$[R]$():
\begin{itemize}
\item output: $out := [R]$
\item exception: none
\end{itemize}

\noindent Param.$P_{tsm}^{dBm}$():
\begin{itemize}
\item output: $out := P_{tsm}^{dBm}$
\item exception: none
\end{itemize}

\noindent Param.$f$():
\begin{itemize}
\item output: $out := f$
\item exception: none
\end{itemize}

\noindent Param.$length_C$():
\begin{itemize}
\item output: $out$ := number of elements in the user-input item $[C]$
\item exception: none
\end{itemize}

\noindent Param.$length_D$():
\begin{itemize}
\item output: $out$ := number of elements in the user-input item $[D]$
\item exception: none
\end{itemize}

\noindent Param.$length_T$():
\begin{itemize}
\item output: $out$ := number of elements in the user-input item $[T]$
\item exception: none
\end{itemize}

\noindent Param.$length_R$():
\begin{itemize}
\item output: $out$ := number of elements in the user-input item $[R]$
\item exception: none
\end{itemize}

\noindent load\_params(s1: string, s2: string, s3: string):
\begin{itemize}
\item transition: \\
The file names s1, s2, and s3 are associated with tsmFile, spFile, and wallsFile respectively.\\
The state variables are modified with the following procedures:\\
1. Read data from the three files to populate the state variables from \ref{SRS-R_Inputs} (from $Pos_{tsm}$ to $f$).\\
2. Store the lengths of $[C], [D], [T], and [R]$ as $length_C, length_D, length_T$, and $length_R$ respectively. \\
3. verify\_params()\\
\item exception: $exc$ := any of the file names (s1, s2, or s3) cannot be found OR of any file's format (tsmFile, spFile, or wallsFile) is incorrect $\Rightarrow$ FileError
\end{itemize}

\noindent verify\_params():
\begin{itemize}
\item output: $out$ := none
\item exception: $exc$ := \\
$\neg(T_{min} \leq T_x \leq T_{max}$ $\forall$ $T_x \in [T]) \Rightarrow$ badTransmittance\\
$\neg(R_{min} \leq R_x \leq R_{max}$ $\forall$ $R_x \in [R]) \Rightarrow$ badReflectance\\
$\neg(P_{min}^{dBm} \leq P_{tsm}^{dBm} \leq P_{max}^{dBm}) \Rightarrow$ badTransmissionPower\\
$\neg(f_{min} \leq f \leq f_{max}) \Rightarrow$ badSignalFrequency\\
$\neg(x_{min} \leq x_{C_x} \leq x_{max}$ $\forall$ $C_x \in [C]) \Rightarrow$ badPosition\\
$\neg(y_{min} \leq y_{C_x} \leq y_{max}$ $\forall$ $C_x \in [C]) \Rightarrow$ badPosition\\
$\neg(x_{min} \leq x_{D_x} \leq x_{max}$ $\forall$ $D_x \in [D]) \Rightarrow$ badPosition\\
$\neg(y_{min} \leq y_{D_x} \leq y_{max}$ $\forall$ $D_x \in [D]) \Rightarrow$ badPosition\\
$\neg(length_C = length_D = length_T = length_R) \Rightarrow$ inconsistentWallParams\\
\end{itemize}

\subsubsection{Local Functions} None



\newpage
\section{MIS of Point Module} \label{pointModule}

\subsection{Module}

Point

\subsection{Uses}

None

\subsection{Syntax}

\subsubsection{Exported Constants}

None

\subsubsection{Exported Access Programs}

\begin{center}
\begin{tabular}{p{4cm} p{3cm} p{2cm} p{4cm}}
\hline
\textbf{Name} & \textbf{In} & \textbf{Out} & \textbf{Exceptions} \\
\hline
create & x\_coordinate: $\mathbb{R}$, y\_coordinate: $\mathbb{R}$ & Point & - \\
set\_coordinates & x\_coordinate: $\mathbb{R}$, y\_coordinate: $\mathbb{R}$ & - & - \\
get\_coordinates & - & $x: \mathbb{R}, y: \mathbb{R}$ & - \\
\hline
\end{tabular}
\end{center}

\subsection{Semantics}

\subsubsection{State Variables}
$x: \mathbb{R}$\\
$y: \mathbb{R}$
\subsubsection{Environment Variables} None

\subsubsection{Assumptions} None

\subsubsection{Access Routine Semantics}

\noindent create(x\_coordinate: $\mathbb{R}$, y\_coordinate: $\mathbb{R}$):
\begin{itemize}
\item transition: \\
$x$ := x\_coordinate\\
$y$ := y\_coordinate
\item output: $out$ := self
\end{itemize}

\noindent set\_coordinates(x\_coordinate: $\mathbb{R}$, y\_coordinate: $\mathbb{R}$):
\begin{itemize}
\item transition: \\
$x$ := x\_coordinate\\
$y$ := y\_coordinate
\item output: none
\end{itemize}

\noindent get\_coordinates():
\begin{itemize}
\item output: $out := x,y$
\end{itemize}

\subsubsection{Local Functions}
None

\newpage
\section{MIS of Wall Module} \label{wallModule}

\subsection{Module}

Wall

\subsection{Uses}

Point (\autoref{pointModule}), EquationFinder (\autoref{equationFinder})

\subsection{Syntax}

\subsubsection{Exported Constants}

None

\subsubsection{Exported Access Programs}

\begin{center}
\begin{tabular}{p{4cm} p{2.3cm} p{3cm} p{3cm}}
\hline
\textbf{Name} & \textbf{In} & \textbf{Out} & \textbf{Exceptions} \\
\hline
create & start: Point, end: Point & Wall & invalidWall \\
set\_start\_point & Point & - & invalidWall \\
set\_end\_point & Point & - & invalidWall \\
get\_start\_point & - & Point & - \\
get\_end\_point & - & Point & - \\
set\_transmittance & $\mathbb{R}$ & - & - \\
set\_reflectance & $\mathbb{R}$ & - & - \\
get\_transmittance & - & $\mathbb{R}$ & - \\
get\_reflectance & - & $\mathbb{R}$ & - \\
get\_unit\_normal & - & $n1: \mathbb{R}$,  $n2: \mathbb{R}$ & - \\
get\_line\_equation & - & $m1: \mathbb{R}$, $m2: \mathbb{R}$, $k: \mathbb{R}$ & - \\
\hline
\end{tabular}
\end{center}

\subsection{Semantics}

\subsubsection{State Variables}
$C_x:$ Point\\
$D_x:$ Point\\
$T_x: \mathbb{R}$\\
$R_x: \mathbb{R}$\\
$m1: \mathbb{R}$\\
$m2: \mathbb{R}$\\
$k: \mathbb{R}$\\
$n1: \mathbb{R}$\\
$n2: \mathbb{R}$

\subsubsection{Environment Variables} None

\subsubsection{Assumptions} None

\subsubsection{Access Routine Semantics}

\noindent create(start: Point, end: Point, transmittance: $\mathbb{R}$, reflectance: $\mathbb{R}$):
\begin{itemize}
\item transition: \\
$C_x$ := start\\
$D_x$ := end\\
$T_x$ := transmittance\\
$R_x$ := reflectance\\

Use Equation Finder Module to find equation parameters:\\
$m1, m2, k$ := EquationFinder.find\_equation(start, end)\\

Find the unit normal vector:\\
$n1, n2$ := find\_unit\_normal($C_x, D_x$)\\
\item output: $out$ := self
\item exception: $exc$ := $C_x$ and $D_x$ have the same coordinates $\Rightarrow$ invalidWall
\end{itemize}

\noindent set\_start\_point(start: Point):
\begin{itemize}
\item transition: \\
$C_x$ := start\\

Use Equation Finder Module to find equation parameters:\\
$m1, m2, k$ := EquationFinder.find\_equation($C_x$, $D_x$)\\

Find the unit normal vector:\\
$n1, n2$ := find\_unit\_normal($C_x, D_x$)\\
\item output: none
\item exception: $exc$ := $C_x$ and $D_x$ have the same coordinates $\Rightarrow$ invalidWall
\end{itemize}

\noindent set\_end\_point(end: Point):
\begin{itemize}
\item transition: \\
$D_x$ := end\\

Use Equation Finder Module to find equation parameters:\\
$m1, m2, k$ := EquationFinder.find\_equation($C_x$, $D_x$)\\

Find the unit normal vector:\\
$n1, n2$ := find\_unit\_normal($C_x, D_x$)\\
\item output: none
\item exception: $exc$ := $C_x$ and $D_x$ have the same coordinates $\Rightarrow$ invalidWall
\end{itemize}

\noindent get\_start\_point():
\begin{itemize}
\item output: $out$ := $C_x$
\item exception: none
\end{itemize}

\noindent get\_end\_point():
\begin{itemize}
\item output: $out$ := $D_x$
\item exception: none
\end{itemize}

\noindent set\_transmittance(transmittance: $\mathbb{R}$):
\begin{itemize}
\item transition: $T_x$ := transmittance
\item output: none
\item exception: none
\end{itemize}

\noindent set\_reflectance(resistance: $\mathbb{R}$):
\begin{itemize}
\item transition: $R_x$ := resistance
\item output: none
\item exception: none
\end{itemize}

\noindent get\_transmittance():
\begin{itemize}
\item output: $out := T$
\item exception: none
\end{itemize}

\noindent get\_reflectance():
\begin{itemize}
\item output: $out := R$
\item exception: none
\end{itemize}

\noindent get\_unit\_normal():
\begin{itemize}
\item output: $out := n1, n2$
\item exception: none
\end{itemize}

\noindent get\_line\_equation():
\begin{itemize}
\item output: $out := m1, m2, k$
\item exception: none
\end{itemize}

\subsubsection{Local Functions}
\noindent  find\_unit\_normal($C_x, D_x$):
\begin{itemize}
\item transition:\\
$x_{C_x}, y_{C_x}$ := $C_x$.get\_coordinates()\\
$x_{D_x}, y_{D_x}$ := $D_x$.get\_coordinates()\\
$\begin{bmatrix} n1 & n2 \end{bmatrix} := \begin{bmatrix}
  (x_{D_x}-x_{C_x}) & (y_{D_x}-y_{C_x})\\
  \end{bmatrix}$
  $\begin{bmatrix}
  0 & -1\\
  1 & 0
  \end{bmatrix} \cdot \frac{1}{\sqrt{(y_{D_x}-y_{C_x})^2+(x_{D_x}-x_{C_x})^2}}
  $
\item output: out := $n1, n2$
\item exception: none
\end{itemize}

\newpage
\section{MIS of Floor Map Module} \label{floorMap}

\subsection{Module}

FloorMap

\subsection{Uses}

Param(\autoref{inputParamModule}), Point(\autoref{pointModule}), Wall(\autoref{wallModule})

\subsection{Syntax}

\subsubsection{Exported Constants}

None

\subsubsection{Exported Access Programs}

\begin{center}
\begin{tabular}{p{4cm} p{2.7cm} p{2cm} p{4cm}}
\hline
\textbf{Name} & \textbf{In} & \textbf{Out} & \textbf{Exceptions} \\
\hline
create & - & FloorMap & - \\
get\_wall & $\mathbb{N}_0$ & Wall & invalidIndex \\
get\_map & - & FloorMap & - \\
\hline
\end{tabular}
\end{center}

\subsection{Semantics}

\subsubsection{State Variables}
wall\_list: set[Wall]\\
wall\_list\_length: $\mathbb{N}_0$\\
\subsubsection{Environment Variables} None

\subsubsection{Assumptions} create() will be called before the FloorMap object can be accessed.

\subsubsection{Access Routine Semantics}

\noindent create():
\begin{itemize}
\item transition: \\
Get parameters of all walls from Input Parameters Module (\autoref{inputParamModule}):\\
Let $list_C$ = Param.$[C]$();\\
Let $list_D$ = Param.$[D]$();\\
Let $list_T$ = Param.$[T]$();\\
Let $list_R$ = Param.$[R]$();\\
Then:\\
wall\_list\_length := Param.$length_C$()\\
Then create wall\_list as such:\\
for $i$ = \{0, 1, 2, ... , (wall\_list\_length - 1)\},\\
\indent\indent wall\_list($i$) := Wall.create($list_C(i), list_D(i), list_T(i), list_R(i)$)
\item output: $out$ := self
\item exception: none
\end{itemize}

\noindent get\_wall(index: $\mathbb{N}_0$):
\begin{itemize}
\item output: $out$ := wall\_list(index)
\item exception: exc := $\neg$(0 $\leq$ index $\leq$ wall\_list\_length - 1) $\Rightarrow$ invalidIndex
\end{itemize}

\noindent get\_map():
\begin{itemize}
\item output: $out$ := self
\item exception: none
\end{itemize}

\subsubsection{Local Functions}
None



\newpage
\section{MIS of Equation Finder} \label{equationFinder}

\subsection{Module}

EquationFinder

\subsection{Uses}
Point(\autoref{pointModule})

\subsection{Syntax}

\subsubsection{Exported Constants}

None

\subsubsection{Exported Access Programs}

\begin{center}
\begin{tabular}{p{4cm} p{2.3cm} p{3cm} p{4cm}}
\hline
\textbf{Name} & \textbf{In} & \textbf{Out} & \textbf{Exceptions} \\
\hline
find\_equation & start: Point, end: Point & m1: $\mathbb{R}$, m2: $\mathbb{R}$, k: $\mathbb{R}$ & - \\
\hline
\end{tabular}
\end{center}

\subsection{Semantics}

\subsubsection{State Variables} None

\subsubsection{Environment Variables} None

\subsubsection{Assumptions} None

\subsubsection{Access Routine Semantics}

\noindent find\_equation(start: Point, end: Point):
\begin{itemize}
\item output:\\
Let $x_{C_x}, y_{C_x}$ = start.get\_coordinates();\\
Let $x_{D_x}, y_{D_x}$ = end.get\_coordinates();\\
Then\\
$ m1 :=
  \begin{cases}
  -\frac{y_{D_x}-y_{C_x}}{x_{D_x}-x_{C_x}} & $if $ x_{D_x}-x_{C_x}\neq{0}\\
  1 & $else$\\
  \end{cases}
  $\\
  $ m2 :=
  \begin{cases}
  1 & $if $ x_{D_x}-x_{C_x}\neq{0}\\
  0 & $else$\\
  \end{cases}
  $\\
  $k := m1\cdot x_{C_x} + m2\cdot y_{C_x} = m1\cdot x_{D_x} + m2\cdot y_{D_x}$\\
  $out := m1, m2, k$
\item exception: none
\end{itemize}

\subsubsection{Local Functions}
None



\newpage
\section{MIS of Linear Signal Path Module} \label{linearPathModule}

\subsection{Module}

LinearPath

\subsection{Uses}

Point (\autoref{pointModule}), EquationFinder (\autoref{equationFinder}) 

\subsection{Syntax}

\subsubsection{Exported Constants} None

\subsubsection{Exported Access Programs}

\begin{center}
\begin{tabular}{p{4cm} p{2.3cm} p{3cm} p{2cm}}
\hline
\textbf{Name} & \textbf{In} & \textbf{Out} & \textbf{Exceptions} \\
\hline
create & start: Point, end: Point & LinearPath & - \\
get\_equation & - & $m1: \mathbb{R}$, $m2: \mathbb{R}$, $k: \mathbb{R}$ & - \\
get\_start\_point & - & Point & - \\
get\_end\_point & - & Point & - \\
get\_length & - & $\mathbb{R}$ & - \\
\hline
\end{tabular}
\end{center}

\subsection{Semantics}

\subsubsection{State Variables}
$E$: Point\\
$F$: Point\\
$m1: \mathbb{R}$\\
$m2: \mathbb{R}$\\
$k: \mathbb{R}$\\
$length: \mathbb{R}$

\subsubsection{Environment Variables} None

\subsubsection{Assumptions} None

\subsubsection{Access Routine Semantics}

\noindent create(start: Point, end: Point):
\begin{itemize}
\item transition: \\
$E$ := start\\
$F$ := end\\

Use Equation Finder Module to find the path's equation parameters:\\
$m1, m2, k$ :=  EquationFinder.find\_equation($E, F$)\\


Let $x_E, y_E$ = $E$.get\_coordinates();\\
Let $x_F, y_F$ = $F$.get\_coordinates();\\
The find the physical length of the linear path:\\
$length := \sqrt{(x_E - x_F)^2 + (y_E - y_F)^2}$
\item output: $out$ := self 
\item exception: none
\end{itemize}

\noindent get\_equation():
\begin{itemize}
\item output: $out :=  m1, m2, k$
\item exception: none
\end{itemize}

\noindent get\_start\_point():
\begin{itemize}
\item output: $out :=  E$
\item exception: none
\end{itemize}

\noindent get\_end\_point():
\begin{itemize}
\item output: $out :=  F$
\item exception: none
\end{itemize}

\noindent get\_length():
\begin{itemize}
\item output: $out :=  length$\
\item exception: none
\end{itemize}

\subsubsection{Local Functions} None



\newpage
\section{MIS of Intersection Module} \label{intersectionModule}

\subsection{Module}

Intersection

\subsection{Uses}

Point (\autoref{pointModule}), Wall (\autoref{wallModule}), LinearPath (\autoref{linearPathModule}) 

\subsection{Syntax}

\subsubsection{Exported Constants} None

\subsubsection{Exported Access Programs}

\begin{center}
\begin{tabular}{p{4cm} p{2.3cm} p{3cm} p{2cm}}
\hline
\textbf{Name} & \textbf{In} & \textbf{Out} & \textbf{Exceptions} \\
\hline
find\_intersection & Wall, LinearPath & Point & - \\
is\_valid & Wall, LinearPath & Boolean & - \\
\hline
\end{tabular}
\end{center}

\subsection{Semantics}

\subsubsection{State Variables}None

\subsubsection{Environment Variables} None

\subsubsection{Assumptions} None

\subsubsection{Access Routine Semantics}

\noindent find\_intersection($wall$: Wall, $path$: LinearPath):
\begin{itemize}
\item output: \\
Let $M$ = 
$\begin{bmatrix}
wall.m1 & wall.m2\\
path.m1 & path.m2
\end{bmatrix}$;\\
Let $K$ = 
$\begin{bmatrix}
wall.k\\
path.k
\end{bmatrix}$;\\

$t'$ := Point.create($x, y$) such that $M
\begin{bmatrix}
x\\
y
\end{bmatrix} = K
$, if $det(M) \neq 0$; or\\

$t'$ := Point.create(0, 0), if $det(M) = 0$.

$out$ := $t'$
\item exception: none
\end{itemize}

\noindent is\_valid($wall$: Wall, $path$: LinearPath):
\begin{itemize}
\item output: \\
Let $M$ = 
$\begin{bmatrix}
wall.m1 & wall.m2\\
path.m1 & path.m2
\end{bmatrix}$;\\
Let $K$ = 
$\begin{bmatrix}
wall.k\\
path.k
\end{bmatrix}$;\\

If $det(M) \neq 0$, \\
$t'$ := Point.create($x, y$) such that $M
\begin{bmatrix}
x\\
y
\end{bmatrix} = K
$\\
If\\ 
$\max(\min(wall.C_x.x, wall.D_x.x),\min(path.C_x.x, path.D_x.x)) < t'.x < min(max(wall.C_x.x, wall.D_x.x),max(path.C_x.x, path.D_x.x))$\\
and\\ 
$\max(\min(wall.C_x.y, wall.D_x.y),\min(path.C_x.y, path.D_x.y)) < t'.x < min(max(wall.C_x.y, wall.D_x.y),max(path.C_x.y, path.D_x.y))$,\\
$Ind$ := 1;\\
otherwise $Ind$ := 0.\\
If $det(M) = 0$,\\
$Ind$ := 0.\\

$out$ := $Ind$
\item exception: none
\end{itemize}


\subsubsection{Local Functions} None



\newpage
\section{MIS of Linear Path Loss Module} \label{linearLossModule}

\subsection{Module}

LinearLoss

\subsection{Uses}
FloorMap (\autoref{floorMap}), LinearPath (\autoref{linearPathModule}), Intersection (\autoref{intersectionModule})

\subsection{Syntax}

\subsubsection{Exported Constants}

None

\subsubsection{Exported Access Programs}

\begin{center}
\begin{tabular}{p{4cm} p{3cm} p{3cm} p{4cm}}
\hline
\textbf{Name} & \textbf{In} & \textbf{Out} & \textbf{Exceptions} \\
\hline
find\_linear\_path\_loss & path:LinearPath, f: $\mathbb{R}$ & $\mathbb{R}$ & - \\
\hline
\end{tabular}
\end{center}

\subsection{Semantics}

\subsubsection{State Variables} None

\subsubsection{Environment Variables} None

\subsubsection{Assumptions} FloorMap.create() will be called before calling FloorMap.get\_map().

\subsubsection{Access Routine Semantics}

\noindent find\_linear\_path\_loss ($path$: LinearPath, $freq:\mathbb{R}$):
\begin{itemize}
\item output: \\
The output is dependent on two parameters: $FSPL$ and $T_{total}$.\\
Update $FSPL$:\\
$map$: FloorMap.get\_map()\\
$FSPL$ := $(\frac{4\pi \cdot path.length}{3\times10^8})^2$\\

Update $T_{total}$:\\
$T_{total}$ := $\Pi_{x=0}^{N_w}(wall_x \cdot T_x^{Ind_{t,x}})$\\
Where\\
$N_w$ := $map$.wall\_list\_length - 1\\
$wall_x$ := map.get\_wall($x$)\\
$Ind_{t,x}$ := Intersection.is\_valid($wall_x, path$)\\

$out$ := $\frac{T_{total}}{FSPL}$
\item exception: none
\end{itemize}

\subsubsection{Local Functions}
None



\newpage
\section{MIS of Line-Of-Sight Signal Module} \label{losModule}

\subsection{Module}

LineOfSight

\subsection{Uses}
Param (\autoref{inputParamModule}), Point (\autoref{pointModule}), LinearPath (\autoref{linearPathModule}),  LinearLoss (\autoref{linearLossModule})

\subsection{Syntax}

\subsubsection{Exported Constants}

$\phi_{LOS}$ := 0

\subsubsection{Exported Access Programs}

\begin{center}
\begin{tabular}{p{4cm} p{3cm} p{3cm} p{4cm}}
\hline
\textbf{Name} & \textbf{In} & \textbf{Out} & \textbf{Exceptions} \\
\hline
create & sampling\_point: Point & LineOfSight & - \\
get\_path\_length & - & $\mathbb{R}$ & - \\
get\_amplitude & - & $\mathbb{R}$ & - \\
get\_phase\_angle & - & $\mathbb{R}$ & - \\
\hline
\end{tabular}
\end{center}

\subsection{Semantics}

\subsubsection{State Variables} 
$Pos_{sp}$: Point\\
$d_{tsm,sp}$: $\mathbb{R}$\\
$P_{LOS}$: $\mathbb{R}$

\subsubsection{Environment Variables} None

\subsubsection{Assumptions} None

\subsubsection{Access Routine Semantics}

\noindent create(sampling\_point: Point):
\begin{itemize}
\item transition: \\
Update $Pos_{sp}$:\\
$Pos_{sp}$ := sampling\_point\\

Update $d_{tsm,sp}$:\\
$freq$ := Param.$f()$\\
$Pos_{tsm}$ := Param.$Pos_{tsm}()$\\
$path$ := LinearPath.create($Pos_{tsm}, Pos_{sp}$)\\
$d_{tsm,sp}$ := $path$.get\_length()\\

Update $P_{LOS}$:\\
$P_{tsm}^{dBm}$ := Param.$P_{tsm}^{dBm}()$\\
$P_{tsm}$ := $10^\frac{P_{tsm}^{dBm}-30}{10}$\\
$P_{LOS}$ := $P_{tsm} \cdot$ LinearLoss.find\_linear\_path\_loss($path, freq$)\\
\item output: $out$ := self
\item exception: none
\end{itemize}

\noindent get\_path\_length():
\begin{itemize}
\item output: $out$ := $d_{tsm,sp}$
\item exception: none
\end{itemize}

\noindent get\_amplitude():
\begin{itemize}
\item output: $out$ := $P_{LOS}$
\item exception: none
\end{itemize}

\noindent get\_phase\_angle():
\begin{itemize}
\item output: $out$ := $\phi_{LOS}$
\item exception: none
\end{itemize}

\subsubsection{Local Functions}
None



\newpage
\section{MIS of Specular Reflection Module} \label{specularReflection}

\subsection{Module}

Specular

\subsection{Uses}
Point (\autoref{pointModule}), Wall (\autoref{wallModule}), LinearPath (\autoref{linearPathModule}),  Intersection (\autoref{intersectionModule})

\subsection{Syntax}

\subsubsection{Exported Constants} None

\subsubsection{Exported Access Programs}

\begin{center}
\begin{tabular}{p{4cm} p{2cm} p{4cm} p{2cm}}
\hline
\textbf{Name} & \textbf{In} & \textbf{Out} & \textbf{Exceptions} \\
\hline
get\_mirrored\_paths & $wall_x$:Wall, $start$:Point, $end$:Point & $path_{RS1}$:LinearPath, $path_{RS2}$:LinearPath, $Ind_{r,x}$: Boolean & - \\

\hline
\end{tabular}
\end{center}

\subsection{Semantics}

\subsubsection{State Variables} None

\subsubsection{Environment Variables} None

\subsubsection{Assumptions} None

\subsubsection{Access Routine Semantics}

\noindent get\_mirrored\_paths($wall_x$:Wall, $start$:Point, $end$:Point):
\begin{itemize}
\item output: \\
Let $n$ := $\begin{bmatrix}
wall_x.n1 & wall_x.n2
\end{bmatrix}$;\\
Let $t$ := $\begin{bmatrix}
wall_x.C_x.x & wall_x.C_x.y
\end{bmatrix}$;\\
Let $p$ := $\begin{bmatrix}
start.x & start.y)
\end{bmatrix}$;\\
Then solve\\
$\begin{bmatrix}
p'_x & p'_y\\
\end{bmatrix} = p-2n(n\cdot(p-t))$\\
for the values of $p'_x$ and $p'_y$.\\
let $p'$ := Point.create($p'_x$, $p'_y$);\\
Let $mirrored\_path$ := LinearPath.create($p',end$);\\
Then\\
$t'$ := Intersection.find\_intersection($wall_x, mirrored\_path$);\\
$Ind_{r,x}$ := Intersection.is\_valid($wall_x, mirrored\_path$);\\
$path_{RS1}$ := LinearPath.create($p,t'$);\\
$path_{RS2}$ := LinearPath.create($t',end$);\\
$out$ := $path_{RS1}$, $path_{RS2}$, $Ind_{r,x}$
\item exception: none
\end{itemize}

\subsubsection{Local Functions}
None



\newpage
\section{MIS of First-Order Reflection Signal Module} \label{forsModule}

\subsection{Module}

FirstOrderReflection

\subsection{Uses}
Param (\autoref{inputParamModule}), Wall (\autoref{wallModule}), LinearLoss (\autoref{linearPathModule}), LineOfSight (\autoref{losModule}), Specular (\autoref{specularReflection})

\subsection{Syntax}

\subsubsection{Exported Constants} None

\subsubsection{Exported Access Programs}

\begin{center}
\begin{tabular}{p{3cm} p{3cm} p{4cm} p{2cm}}
\hline
\textbf{Name} & \textbf{In} & \textbf{Out} & \textbf{Exceptions} \\
\hline
create & $wall_x$: Wall, $sampling\_point$: Point, $LOS$: LineOfSight & FirstOrderReflection & - \\
get\_amplitude & - & $\mathbb{R}$ & - \\
get\_phase\_angle & - & $\mathbb{R}$ & - \\

\hline
\end{tabular}
\end{center}

\subsection{Semantics}

\subsubsection{State Variables} 
$Pos_{sp}$: Point\\
$path_{RS1}$: LinearPath\\
$path_{RS2}$: LinearPath\\
$Ind_{r,x}$: Boolean\\
$P_{FORS}$: $\mathbb{R}$\\
$\phi_{FORS}$: $\mathbb{R}$\\

\subsubsection{Environment Variables} None

\subsubsection{Assumptions} None

\subsubsection{Access Routine Semantics}

\noindent create($wall_x$:Wall, sampling\_point: Point, $LOS$: LineOfSight):
\begin{itemize}
\item transition:\\
Update $Pos_{sp}$:\\
$Pos_{sp}$ := $sampling\_point$\\

Update $path_{RS1}$, $path_{RS2}$, and $Ind_{r,x}$:\\
$freq$ := Param.$f()$\\
$Pos_{tsm}$ := Param.$Pos_{tsm}()$\\
$path_{RS1}$, $path_{RS2}$, $Ind_{r,x}$ := Specular.get\_mirrored\_paths($wall_{x}, Pos_{tsm}, Pos_{sp}$)\\

Update $P_{FORS}$:\\
If $Ind_{r,x}$ = 1:\\
$P_{tsm}^{dBm}$ := Param.$P_{tsm}^{dBm}()$\\
$P_{tsm}$ := $10^\frac{P_{tsm}^{dBm}-30}{10}$\\
$transmittance_{RS1}$ := LinearLoss.find\_linear\_path\_loss($path_{RS1}, freq$)\\
$transmittance_{RS2}$ := LinearLoss.find\_linear\_path\_loss($path_{RS2}, freq$)\\
$P_{FORS}$ := $P_{tsm} \cdot transmittance_{RS1} \cdot transmittance_{RS2} \cdot wall_x.$get\_reflectance()\\
Otherwise:\\
$P_{FORS} := 0$\\

Update $\phi_{FORS}$:\\
If $Ind_{r,x}$ = 1:\\
$\phi_{FORS} := 2 \pi f \frac{path_{RS1}.length + path_{RS2}.length - LOS.d_{tsm,sp}}{3\times10^8}$
Otherwise:\\
$\phi_{FORS} := 0$

\item output: $out$ := self
\item exception: none
\end{itemize}

\noindent get\_amplitude():
\begin{itemize}
\item output: $out$ := $P_{FORS}$
\item exception: none
\end{itemize}

\noindent get\_phase\_angle():
\begin{itemize}
\item output: $out$ := $\phi_{FORS}$
\item exception: none
\end{itemize}

\subsubsection{Local Functions}
None



\newpage
\section{MIS of Received Signal Strength Module} \label{rssModule}

\subsection{Module}

ReceivedSignalStrength

\subsection{Uses}
FloorMap (\autoref{floorMap}), LineOfSight (\autoref{losModule}), FirstOrderReflection (\autoref{forsModule})

\subsection{Syntax}

\subsubsection{Exported Constants} None

\subsubsection{Exported Access Programs}

\begin{center}
\begin{tabular}{p{4cm} p{4cm} p{2cm} p{2cm}}
\hline
\textbf{Name} & \textbf{In} & \textbf{Out} & \textbf{Exceptions} \\
\hline
get\_received\_power & $map$: FloorMap, $sampling\_point$: Point & $\mathbb{R}$ & invalidReceivedStrength \\

\hline
\end{tabular}
\end{center}

\subsection{Semantics}

\subsubsection{State Variables} None

\subsubsection{Environment Variables} None

\subsubsection{Assumptions} FloorMap.create() will be called before calling FloorMap.get\_map() in this module.

\subsubsection{Access Routine Semantics}

\noindent get\_received\_power($map$: FloorMap, $sampling\_point$: Point):
\begin{itemize}
\item output:\\
Get Floor Map:\\
$map$ := FloorMap.get\_map()\\
$map\_complexity$ := $map$.wall\_list\_length - 1

Find Line-Of-Sight Signal\\
$LOS$ := LineOfSight.create($sampling\_point$)\\

Find First-Order reflection signals by wall:\\
For $x$ in (0, 1, 2, ... , $map\_complexity$):\\
$wall_x$ := $map$.get\_wall($x$)\\
$FORS(x)$ := FirstOrderReflection.create($wall_x, sampling\_point, LOS$)\\

Find total received signal:\\
$P_{LOS} := LOS.$get\_amplitude()\\
$\phi_{LOS} := LOS.$get\_phase\_angle()\\
$P_{FORS_x} := FORS(x).$get\_amplitude()\\
$\phi_{FORS_x} := FORS(x).$get\_phase\_angle()\\
$P_{sp}\angle\phi_{sp} := P_{LOS}\angle\phi_{LOS} + \sum_{x = 0}^{map\_complexity}P_{FORS_x}\angle\phi_{FORS_x}$\\
$P_{sp}^{dBm} := 30 + 10\log_{10}(P_{sp})$\\
$out$ := $P_{sp}^{dBm}$
\item exception: $exc$ := $(P_{tsm}^{dBm} \leq P_{sp}^{dBm}) \Rightarrow$ invalidReceivedStrength
\end{itemize}

\subsubsection{Local Functions}
None



\newpage
\section{MIS of Output Module} \label{outputModule}

\subsection{Module}

Output

\subsection{Uses}
Param (\autoref{inputParamModule}), Point (\autoref{pointModule})

\subsection{Syntax}

\subsubsection{Exported Constants} None

\subsubsection{Exported Access Programs}

\begin{center}
\begin{tabular}{p{3cm} p{3cm} p{3cm} p{3cm}}
\hline
\textbf{Name} & \textbf{In} & \textbf{Out} & \textbf{Exceptions} \\
\hline
output & fname: string, $[Pos_{sp}]$: set[Point], $[P_{sp}^{dBm}]$: set[$\mathbb{R}$] & - & - \\

\hline
\end{tabular}
\end{center}

\subsection{Semantics}

\subsubsection{State Variables} None

\subsubsection{Environment Variables} file: a text file

\subsubsection{Access Routine Semantics}

\noindent output(fname, $[Pos_{sp}]$, $[P_{sp}^{dBm}]$):
\begin{itemize}
\item transition: write to environment variable named fname the calculated received signal strengths $[P_{sp}^{dBm}]$ and their corresponding sampling points in $[Pos_{sp}]$.\\
Each line of the output file will be 3 numbers separated by comma: [$Pos_{sp}.x, Pos_{sp}.y, P_{sp}^{dBm}$].

\item exception: none
\end{itemize}

\subsubsection{Local Functions}
None



\newpage
\section{MIS of Specification Parameters} \label{specParamModule}

\subsection{Module}

SpecParam

\subsection{Uses} None

\subsection{Syntax}

\subsubsection{Exported Constants}

From Table \ref{TblSpecParams} in SRS:\\
  $P_{max}^{dBm}$ := 15 (\si{dBm})\\
  $P_{min}^{dBm}$ := -30 (\si{dBm})\\
  $f_{min}$ := 30 (\si{\hertz})\\
  $f_{max}$ := $3\times 10^{11}$  (\si{\hertz})\\
  $x_{min}$ := -20 (\si{\meter})\\
  $x_{max}$ := 20 (\si{\meter})\\
  $y_{min}$ := -20 (\si{\meter})\\
  $y_{max}$ := 20 (\si{\meter})\\

\subsection{Semantics}
N/A



\newpage

\bibliographystyle {plainnat}
\bibliography {../../../refs/References}

\end{document}