\documentclass[12pt, titlepage]{article}

\usepackage{booktabs}
\usepackage{longtable}
\usepackage{tabularx}
\usepackage{hyperref}
\usepackage{pdflscape}
\hypersetup{
    colorlinks,
    citecolor=blue,
    filecolor=black,
    linkcolor=red,
    urlcolor=blue
}
\usepackage[round]{natbib}

%% Comments

\usepackage{color}

\newif\ifcomments\commentstrue

\ifcomments
\newcommand{\authornote}[3]{\textcolor{#1}{[#3 ---#2]}}
\newcommand{\todo}[1]{\textcolor{red}{[TODO: #1]}}
\else
\newcommand{\authornote}[3]{}
\newcommand{\todo}[1]{}
\fi

\newcommand{\wss}[1]{\authornote{blue}{SS}{#1}} 
\newcommand{\wxz}[1]{\authornote{cyan}{XZ}{#1}} 
\newcommand{\plt}[1]{\authornote{magenta}{TPLT}{#1}} %For explanation of the template
\newcommand{\an}[1]{\authornote{cyan}{Author}{#1}}

%% Common Parts

\newcommand{\progname}{ProgName} % PUT YOUR PROGRAM NAME HERE %Every program
                                % should have a name


\begin{document}

\title{Project Title: System Verification and Validation Plan for Radio Signal Strength Calculator} 
\author{Xingzhi Liu}
\date{\today}
	
\maketitle

\pagenumbering{roman}

\section{Revision History}

\begin{tabularx}{\textwidth}{p{3cm}p{2cm}X}
\toprule {\bf Date} & {\bf Version} & {\bf Notes}\\
\midrule
Oct 30, 2020 & 1.0 & First Draft\\
Dec 20, 2020 & 1.1 & Revision 1\\
\bottomrule
\end{tabularx}

\newpage

\tableofcontents

\listoftables

\listoffigures

\newpage

\section{Symbols, Abbreviations and Acronyms}

\renewcommand{\arraystretch}{1.2}
\begin{tabular}{l l} 
  \toprule		
  \textbf{symbol} & \textbf{description}\\
  \midrule 
  FR & Functional Requirement\\
  MG & Module Guide\\
  MIS & Module Interface Specification\\
  NFR & Non-Functional Requirement\\
  RSSC & Sadio SignalStrength Calculator\\
  SRS & Software Requirement Specification\\
  T & Test\\
  VV & Validation and Verification\\
  \bottomrule
\end{tabular}\\

\newpage

\pagenumbering{arabic}

This document records and presents the verification and validation plan for RSSC
to help ensure the program meets the requirements. General information including
a quick recall of RSSC's background is given in \autoref{GeneralInfo}. 
\autoref{plan} provides a plan for verification and \autoref{systd} describes 
the system tests, including tests for functional requirements and tests for non-
functional requirements.


\section{General Information}
\label{GeneralInfo}

\subsection{Summary}

The software to test is Radio Signal Strength Calculator (RSSC). The purpose of
RSSC is to simulate the radio signal propagation in the indoor environment defined
by the user, and find the analytical signal strength for the user. 

\subsection{Objectives}

The objective of this document is to build confidence in the software correctness.
To reach this objective, all functional and non-functional requirements will be
tested following the descriptions in this document. 

\subsection{Relevant Documentation}

\begin{itemize}
	\item 
	\href{https://github.com/XingzhiMac/CAS741-Proj/blob/master/docs/SRS/SRS.pdf}
	{SRS} 
\end{itemize}

\section{Plan}\label{plan}
	
\subsection{Verification and Validation Team}

\begin{itemize}
	\item Author: Xingzhi Liu
\end{itemize}

\begin{itemize}
	\item Primary Reviewer: Siddharth (Sid) Shinde
\end{itemize}

\begin{itemize}
	\item Reviewer: Leila Mousapour
\end{itemize}

\begin{itemize}
	\item Reviewer: Shayan Mousavi Masouleh
\end{itemize}

\begin{itemize}
	\item Dr. Spencer Smith
\end{itemize}

\subsection{SRS Verification Plan}

SRS will be done by team members reviewing the document. Team members can put
any comments, suggestions or questions in RSSC's Github repository as issues. The
SRS reviewing team includes the following members: domain expert Siddharth (Sid) Shinde, 
the SRS reviewer Leila Mousapour, Dr. Spencer Smith, and the author Xingzhi Liu.
The author will respond to the issues and make modifications when needed.

\subsection{Design Verification Plan}

Design verification will be done by team members, by reviewing whether the steps 
of calculation in the software follows the physical model in SRS or not. 

\subsection{Implementation Verification Plan}

Implementation verification will be done by testing all the functional and non-functional requirements. Descriptions of the tests can be found in \autoref{tfr} and \autoref{tnfr}. 
In addition, we will undergo static verification by checking
all the codes we build with Pylint. We will also conduct unit testing for modules
within the testing scope. Details for unit testing can be found in \autoref{unittest}.


\subsection{Automated Testing and Verification Tools}
\begin{itemize}
	\item Python Unittest
\end{itemize}
\begin{itemize}
	\item Pylint
\end{itemize}

\subsection{Software Validation Plan}

There are no plans for validation. We may be able to set up a simulation of radio
frequency signal on existing physics simulation tools and take the simulation 
result as a reference to evaluate RSSC's output, but different tools makes 
different assumptions to the indoor signal propagation model and we cannot 
change their assumptions. It is highly unlikely that we could find a tool 
that makes the same assumptions as RSSC. Therefore we do not have any reference 
to evaluate RSSC's correctness.


\section{System Test Description}
\label{systd}
	
\subsection{Tests for Functional Requirements}
\label{tfr}

Functional requirements for RSSC are given in 
\href{https://github.com/XingzhiMac/CAS741-Proj/blob/master/docs/SRS/SRS.pdf}
{SRS} section 5.1. There are 5 functional requirements for RSSC, from R1 to R5.
R1 and R2 are corresponding to inputs, while R3 to R5 are corresponding to outputs.
\autoref{area1} describes the input tests for R1 and R2; and \autoref{area2} 
describes the output tests for R3, R4 and R5.

\subsubsection{Input}\label{Ainput}
\label{area1}

This test verifies the following requirements: \\
\indent  R1: RSSC takes input from the user;\\
\indent  R2: RSSC verifies user inputs.\\
		
\paragraph{Input Tests}

\begin{enumerate}

\item{Valid inputs\\}

Control: Manual
					
Initial State: Pending Input
					
Input:
$Pos_{tsm} = [0,0];$\\
$[Pos_{sp}] = [[10,0],[-10,0]];$\\
$[C] = [[1,1],[2,2],[1,3]];$\\
$[D] = [[2,2],[1,3],[1,1]];$\\
$[T] = [0.1,0.1,0.1];$\\
$[R] = [0.6,0.6,0.6];$\\
$P_{tsm}^{dBm} = 0;$\\
$f = 2.48\times10^{9};$\\

Output: Return an input success message in command line.

Test Case Derivation: RSSC correctly takes the inputs from the user
					
How test will be performed: Tester manually feeds the inputs into
RSSC and executes RSSC.
					
					
\item{Inconsistent input array sizes\\}

Control: Manual
					
Initial State: Pending Input
					
Input: 
$Pos_{tsm} = [0,0];$\\
$[Pos_{sp}] = [[10,0],[-10,0]];$\\
$[C] = [[1,1],[2,2],[1,3]];$\\
$[D] = [[2,2],[1,3]];$\\
$[T] = [0.1,0.1,0.1];$\\
$[R] = [0.6,0.6,0.6];$\\
$P_{tsm}^{dBm} = 0;$\\
$f = 2.48\times10^{9};$\\
					
Output: Return error of inconsistent array size in command line.

Test Case Derivation: Correct error message displays.

How test will be performed: Tester manually feeds the inputs into
RSSC and executes RSSC.


\item{Out-of-range position coordinates\\}

Control: Manual
					
Initial State: Pending Input
					
Input:
$Pos_{tsm} = [21,0];$\\
$[Pos_{sp}] = [[10,0],[-10,0]];$\\
$[C] = [[1,1],[2,2],[1,3]];$\\
$[D] = [[2,2],[1,3],[1,1]];$\\
$[T] = [0.1,0.1,0.1];$\\
$[R] = [0.6,0.6,0.6];$\\
$P_{tsm}^{dBm} = 0;$\\
$f = 2.48\times10^{9};$\\

Output: Return error of out-of-range position coordinates.

Test Case Derivation: Correct error message displays.
					
How test will be performed: Tester manually feeds the inputs into
RSSC and executes RSSC.


\item{Out-of-range transmitter power level\\}

Control: Manual
					
Initial State: Pending Input
					
Input:
$Pos_{tsm} = [0,0];$\\
$[Pos_{sp}] = [[10,0],[-10,0]];$\\
$[C] = [[1,1],[2,2],[1,3]];$\\
$[D] = [[2,2],[1,3],[1,1]];$\\
$[T] = [0.1,0.1,0.1];$\\
$[R] = [0.6,0.6,0.6];$\\
$P_{tsm}^{dBm} = 20;$\\
$f = 2.48\times10^{9};$\\

Output: Return error of out-of-range transmitter power level.

Test Case Derivation: Correct error message displays.
					
How test will be performed: Tester manually feeds the inputs into
RSSC and executes RSSC.


\item{Out-of-range signal frequency\\}

Control: Manual
					
Initial State: Pending Input
					
Input:
$Pos_{tsm} = [0,0];$\\
$[Pos_{sp}] = [[10,0],[-10,0]];$\\
$[C] = [[1,1],[2,2],[1,3]];$\\
$[D] = [[2,2],[1,3],[1,1]];$\\
$[T] = [0.1,0.1,0.1];$\\
$[R] = [0.6,0.6,0.6];$\\
$P_{tsm}^{dBm} = 20;$\\
$f = 2.48\times10^{12};$\\

Output: Return error of out-of-range signal frequency.

Test Case Derivation: Correct error message displays.
					
How test will be performed: Tester manually feeds the inputs into
RSSC and executes RSSC.

\end{enumerate}

\subsubsection{Output}\label{Aoutput}
\label{area2}

This test verifies the following requirements: \\
\indent  R3: RSSC shall find $P_{sp}^{dBm}$;\\
\indent  R4: RSSC shall verify $P_{sp}^{dBm}$;\\
\indent  R5: RSSC shall generate a file to store $P_{sp}^{dBm}$;\\

\paragraph{Output Test}

\begin{enumerate}

\item{Simple single valid output\\}

Control: Manual
					
Initial State: Pending Input
					
Input:
$Pos_{tsm} = [0,0];$\\
$[Pos_{sp}] = [[1,0]];$\\
$[C] = [ ];$\\
$[D] = [ ];$\\
$[T] = [ ];$\\
$[R] = [ ];$\\
$P_{tsm}^{dBm} = 0;$\\
$f = 2.48\times10^{9};$\\

Output: A .txt or .csv file with 1 line, showing the sampling point
position (1,0) and $P_{sp}^{dBm}$ that satisfies $P_{sp}^{dBm} 
\leq 0$. 

Test Case Derivation: RSSC returns the correct output.
					
How test will be performed: Tester manually feeds the inputs into
RSSC and executes RSSC.
					

\item{Simple multiple valid output\\}

Control: Manual
					
Initial State: Pending Input
					
Input:
$Pos_{tsm} = [0,0];$\\
$[Pos_{sp}] = [[1,0],[1,1]];$\\
$[C] = [ ];$\\
$[D] = [ ];$\\
$[T] = [ ];$\\
$[R] = [ ];$\\
$P_{tsm}^{dBm} = 0;$\\
$f = 2.48\times10^{9};$\\

Output: A .txt or .csv file with 2 lines. Line 1 shows the sampling point
position (1,0) and $P_{sp}^{dBm}$ that satisfies $P_{sp}^{dBm} 
\leq 0$. Line 2 shows the sampling point position (1,1) and $P_{sp}^{dBm}$ 
that satisfies $P_{sp}^{dBm} \leq 0$. 

Test Case Derivation: RSSC returns the correct output.
					
How test will be performed: Tester manually feeds the inputs into
RSSC and executes RSSC.

\end{enumerate}

\subsection{Tests for Non-Functional Requirements}
\label{tnfr}

Non-functional requirements are given in SRS section 5.2. There are 5 non-
functional requirements: Portable, Maintainable, and Understandable. The 
rest of this section provides detailed descriptions on how to test them.

\subsubsection{Portable}\label{Aport}

\paragraph{Portability Test} RSSC shall be able to run on different OS. 
We will test execute RSSC on both Windows and Linux Ubuntu.

\begin{enumerate}

\item{Portability on Windows 10\\}

Type: Manual
					
Initial State: Pending Input
					
Input: $Pos_{tsm} = [0,0];$\\
$[Pos_{sp}] = [[1,0]];$\\
$[C] = [ ];$\\
$[D] = [ ];$\\
$[T] = [ ];$\\
$[R] = [ ];$\\
$P_{tsm}^{dBm} = 0;$\\
$f = 2.48\times10^{9};$\\
					
Output: A .txt or .csv file with 1 line, showing the sampling point
position (1,0) and $P_{sp}^{dBm}$ that satisfies $P_{sp}^{dBm} 
\leq 0$. 
					
How test will be performed: Tester deploys RSSC on a machine with Windows 
10 OS and execute with the input set above. On success, RSSC should
provide the output as described.


\item{Portability on Linux Ubuntu\\}

Type: Manual
					
Initial State: Pending Input
					
Input: $Pos_{tsm} = [0,0];$\\
$[Pos_{sp}] = [[1,0]];$\\
$[C] = [ ];$\\
$[D] = [ ];$\\
$[T] = [ ];$\\
$[R] = [ ];$\\
$P_{tsm}^{dBm} = 0;$\\
$f = 2.48\times10^{9};$\\
					
Output: A .txt or .csv file with 1 line, showing the sampling point
position (1,0) and $P_{sp}^{dBm}$ that satisfies $P_{sp}^{dBm} 
\leq 0$. 
					
How test will be performed: Tester deploys RSSC on a virtual machine 
with Ubuntu 20.04 and execute with the input set above. On success, 
RSSC should provide the output as described.

\end{enumerate}

\subsubsection{Maintainable}\label{Amaintain}

\paragraph{Maintainability Test} Proper documents should be included
in this project. 

\begin{enumerate}
\item{Maintainability Test\\}

Type: Manual
					
Initial State: none
					
Input: none
					
Output: none
					
How test will be performed: Tester manually checks the contents in the
Github repo. On success, documents shall be uploaded following the schedule
of CAS741 and no issue shall be closed without a proper response.

\end{enumerate}

\subsubsection{Understandable}\label{Audst}

\paragraph{Understandability Test} Programs of RSSC should be organized,
well commented, and easy to understand.

\begin{enumerate}
\item{Understandability Test\\}

Type: Manual
					
Initial State: none
					
Input: none
					
Output: none
					
How test will be performed: Tester review the code and complete the survey
shown in \autoref{ut}.

\end{enumerate}

\begin{longtable}[h]{l l}
\label{ut}
	\begin{tabular}{l l} 
	\toprule		
	\textbf{Item} & \textbf{Score}\\
	\midrule 
	Variable names are rational and follow consistent conventions & \{0 - 5\}\\
	Functions are well commented on what they do & \{0 - 5\} \\
	Functions tasks are broken down / No super complicated functions & \{0 - 5\} \\
	No repeated chunks in the code & \{0 - 5\} \\
	Code is well organized and follows the order of instance models & \{0 - 5\} \\
	\bottomrule
	\caption{Understandability Survey} \label{Undgradesheet} \\
\end{tabular}\\
\end{longtable}

\subsection{Traceability Between Test Cases and Requirements}

\begin{table}[h!]
	\centering
	\begin{tabular}{|c|c|c|c|c|c|c|c|c|}
		\hline
		& R1 & R2 & R3 & R4 & R5 & Portable & Maintainable & Understandable\\
		\hline
		\nameref{Ainput}    &X&X& & & & & & \\ \hline
		\nameref{Aoutput}   & & &X&X&X& & & \\ \hline
		\nameref{Aport}     & & & & & &X& & \\ \hline
		\nameref{Amaintain} & & & & & & &X& \\ \hline
		\nameref{Audst}     & & & & & & & &X\\ \hline
	\end{tabular}
	\caption{Traceability Between Test Cases and Requirements}
	\label{Table:A_trace}
\end{table}

\section{Unit Test Description}
\label{unittest}

\subsection{Unit Testing Scope}

We will conduct unit testings for the following modules:
\begin{itemize}
	\item Point
	\item Wall
	\item Linear Path
	\item Intersection
	\item Linear Path Loss
	\item Specular Reflection
\end{itemize}
Modules including Control, Input Parameters, Output, and Floor Map have low priorities for
verification than others because their correctness can be adequately proven by 
\nameref{area1} and \nameref{area2}.\\
Module Received Signal Strength only contains some basic numerical calculations so
we do not need to test it separately. \nameref{area2} will be sufficient to verify its functionality.\\
Module Line-Of-Sight Signal and module First-Order Reflection Signal require module Input Parameters to run first before they work, and cannot be verified separately. However, testings in \nameref{area2} requires these two modules working properly to provide outputs. Therefore, these two modules will be verified by \nameref{area2}.

\subsection{Tests for Functional Requirements}

\subsubsection{Upoint}
\label{Point}

This section tests the functionality of Point module, including creating a point
and reading the coordinates of a point.

\begin{enumerate}

\item{point-1\\}

Type: manual
					
Initial State: none
					
Input: \\
x = 0, y = 0;\\
x = 3, y = 4;\\
x = 1, y = 10;\\
x = 3, y = 0;\\
					
Output: 4 points successfully created and their positions printed. Positions of the 4 points should be (0, 0), (3, 4), (1, 10), and (3, 0).\\

Test Case Derivation: A point is created with the input position.\\

How test will be performed: Call function Point(x, y) 4 times with inputs listed above.
Every time after Point() is called, run function get$\_$coordinates() and print the result
as well as the expected result.
    
\end{enumerate}

\subsubsection{Uwall}
\label{wall}

This section tests the functionality of Wall module, including creating a wall and finding the equation and the unit normal vector of itself.

\begin{enumerate}

\item{wall-1\\}

Type: manual
					
Initial State: none
					
Input: \\
a = Point (0, 0);\\
b = Point (1, 10);\\
T = 0.1;\\
R = 0.5;\\
					
Output: \\
m1 = -10;\\
m2 = 1;\\
k = 0;\\
$\left[n1 , n2\right]$ = [ 0.9950, -0.0995 ] or [ -0.9950, 0.0995 ]\\

Test Case Derivation: The equation and unit normal vector are calculated with models shown in SRS.\\

How test will be performed: Call function Wall(a, b, T, R) and print parameters m1, m2, k, n1, and  n2 of the wall object, together with their expected values.

\item{wall-2\\}

Type: manual
					
Initial State: none
					
Input: \\
a = Point (3, 4);\\
b = Point (1, 10);\\
T = 0.1;\\
R = 0.5;\\
					
Output: \\
m1 = 3;\\
m2 = 1;\\
k = 13;\\
$\left[n1 , n2\right]$ = [0.9487, 0.3162] or [-0.9487, -0.3162]\\

Test Case Derivation: The equation and unit normal vector are calculated with models shown in SRS.\\

How test will be performed: Call function Wall(a, b, T, R) and print parameters m1, m2, k, n1, and  n2 of the wall object, together with their expected values.

\item{wall-3\\}

Type: manual
					
Initial State: none
					
Input: \\
a = Point (0, 0);\\
b = Point (3, 4);\\
T = 0.1;\\
R = 0.5;\\
					
Output: \\
m1 = -1.3333;\\
m2 = 1;\\
k = 0;\\
$\left[n1 , n2\right]$ = [0.8, -0.6] or [-0.8, 0.6]\\

Test Case Derivation: The equation and unit normal vector are calculated with models shown in SRS.\\

How test will be performed: Call function Wall(a, b, T, R) and print parameters m1, m2, k, n1, and  n2 of the wall object, together with their expected values.

\item{wall-4\\}

Type: manual
					
Initial State: none
					
Input: \\
a = Point (3, 4);\\
b = Point (3, 0);\\
T = 0.1;\\
R = 0.5;\\
					
Output: \\
m1 = 1;\\
m2 = 0;\\
k = 3;\\
$\left[n1 , n2\right]$ = [1, 0] or [-1, 0]\\

Test Case Derivation: The equation and unit normal vector are calculated with models shown in SRS.\\

How test will be performed: Call function Wall(a, b, T, R) and print parameters m1, m2, k, n1, and  n2 of the wall object, together with their expected values.
    
\end{enumerate}

\subsubsection{Ulp}
\label{LP}

This section tests the functionality of Linear Path module, including creating a Linear Path and finding the equation and the length of itself.

\begin{enumerate}

\item{lp-1\\}

Type: manual
					
Initial State: none
					
Input: \\
a = Point (-2, 2);\\
b = Point (13, 2);\\
					
Output: \\
m1 = 0;\\
m2 = 1;\\
k = 2;\\
length = 15\\

Test Case Derivation: The equation is calculated with the linear equation model shown in SRS. The length is the Euclidean distance between the two input points.

How test will be performed: Call function LinearPath(a, b) and print parameters m1, m2, k, and  length of the LinearPath object, together with their expected values.

\item{lp-2\\}

Type: manual
					
Initial State: none
					
Input: \\
a = Point (-2, 2);\\
b = Point (4, 5);\\
					
Output: \\
m1 = -0.5;\\
m2 = 1;\\
k = 3;\\
length = 6.7082\\

Test Case Derivation: The equation is calculated with the linear equation model shown in SRS. The length is the Euclidean distance between the two input points.

How test will be performed: Call function LinearPath(a, b) and print parameters m1, m2, k, and  length of the LinearPath object, together with their expected values.

\item{lp-3\\}

Type: manual
					
Initial State: none
					
Input: \\
a = Point (-2, 2);\\
b = Point (-2, 10);\\
					
Output: \\
m1 = 1;\\
m2 = 0;\\
k = -2;\\
length = 8\\

Test Case Derivation: The equation is calculated with the linear equation model shown in SRS. The length is the Euclidean distance between the two input points.

How test will be performed: Call function LinearPath(a, b) and print parameters m1, m2, k, and  length of the LinearPath object, together with their expected values.
    
\end{enumerate}

\subsubsection{Uint}
\label{intersection}

This section tests the functionality of Intersection module in finding the position and validity of the potential intersection point of any two line segments given.

\begin{enumerate}

\item{inter-1\\}

Type: manual
					
Initial State: none
					
Input: \\
seg1 = output of test case wall-1 in \nameref{wall};\\
seg2 = output of test case lp-1 in \nameref{LP};\\
					
Output: \\
Intersection position = (0.2, 2.0)\\
validity = 1\\

Test Case Derivation: seg1 is a line segment from (0, 0) to (1, 10); seg2 is a line segment from (-2, 2) to (13, 2). Both of them cross the point (0.2, 2.0), therefore (0.2, 2.0) is a valid intersection of seg1 and seg2.\\

How test will be performed: Call Intersection.find$\_$intersection(seg1, seg2) and print the result
as well as the expected result. Then Call Intersection.is$\_$valid(seg1, seg2) and print the result
as well as the expected result.

\item{inter-2\\}

Type: manual
					
Initial State: none
					
Input: \\
seg1 = output of test case wall-4 in \nameref{wall};\\
seg2 = output of test case lp-3 in \nameref{LP};\\
					
Output: \\
Intersection position = (not important)\\
validity = 0\\

Test Case Derivation: seg1 is a line segment from (3, 4) to (3, 0); seg2 is a line segment from (-2, 2) to (-2, 10). seg1 and seg2 are parallel, so they do not have intersection points. Therefore the validity is 0 and the position of the potential intersection point is unessential.\\

How test will be performed: Call Intersection.find$\_$intersection(seg1, seg2) and print the result
as well as the expected result. Then Call Intersection.is$\_$valid(seg1, seg2) and print the result
as well as the expected result.

\item{inter-3\\}

Type: manual
					
Initial State: none
					
Input: \\
seg1 = output of test case wall-3 in \nameref{wall};\\
seg2 = output of test case lp-3 in \nameref{LP};\\
					
Output: \\
Intersection position = (-2, -2.667)(not important)\\
validity = 0\\

Test Case Derivation: seg1 is a line segment from (0, 0) to (3, 4); seg2 is a line segment from (-2, 2) to (-2, 10). extension lines of the two segments intersect at (-2, -2.667), but the two line segments do not cross that point. Therefore the potential intersection point is invalid.\\

How test will be performed: Call Intersection.find$\_$intersection(seg1, seg2) and print the result
as well as the expected result. Then Call Intersection.is$\_$valid(seg1, seg2) and print the result
as well as the expected result.
    
\end{enumerate}

\subsubsection{Ull}
\label{ll}

This section tests the functionality of Linear Path Loss module to determine the power loss of a signal along a given linear path.

\begin{enumerate}

\item{ll-1\\}

Type: manual
					
Initial State: none
					
Input: \\
path1 = output of test case lp-1 in \nameref{LP};\\
frequency = $2.48\times10^9$;\\
floor$\_$map = a FloorMap object with the following3 walls:\\
\indent wall1 = output of test case wall-1 in \nameref{wall};\\
\indent wall2 = output of test case wall-2 in \nameref{wall};\\
\indent wall2 = output of test case wall-3 in \nameref{wall};\\
					
Output: linear path loss = $4.1185\times10^-09$.\\

Test Case Derivation: The linear path loss is calculated with the Line Of Sight Signal Strength model described in SRS.\\

How test will be performed: Call function LinearLoss.find$\_$linear$\_$path$\_$loss(path1, frequency, floor$\_$map) and print the result as well as the expected result.
    
\end{enumerate}

\subsubsection{Uspec}
\label{specular}

This section tests the functionality of Specular Reflection in finding the validity and potential path of a potential first-order reflected signal.

\begin{enumerate}

\item{spec-1\\}

Type: manual
					
Initial State: none
					
Input: \\
wall = output of test case wall-1 in \nameref{wall};\\
start = Point(-2, 2);\\
end = Point(-1, 3);\\
					
Output: \\
first half of the signal path (path1) = a path from Point(-2, 2) to Point(0.01884, 0.1884)\\
first half of the signal path (path1) = a path from Point(0.01884, 0.1884) to Point(-1, 3)\\
validity = 1\\

Test Case Derivation: the reflected image of the start point (-2, 2) against the given wall is at (2.3564, 1.5644). The virtual signal path from the reflected image of the starting point to the ending point intersects the wall at Point(0.1782, 1.782), which is a valid intersection. These information implies the existence of a first-order reflected signal from Point(-2, 2) to Point(-1, 3), reflected by the input wall at Point(0.1782, 1.782). The module's task is to find such a signal.\\

How test will be performed: Call Specular.get$\_$mirrored$\_$paths(wall, start, end)) and print the output as well as the expected results.

\item{spec2\\}

Type: manual
					
Initial State: none
					
Input: \\
wall = output of test case wall-2 in \autoref{wall};\\
start = Point(-2, 2);\\
end = Point(-1, 3);\\
					
Output: \\
validity = 0\\

Test Case Derivation: the reflected image of the start point (-2, 2) against the given wall is at (8.2, 5.4). The virtual signal path from the reflected image of the starting point to the ending point is supposed to intersect the wall at Point(3.1, 3.7), but this is not a valid intersection. These information implies that there is no first-order reflected signals from Point(-2, 2) to Point(-1, 3).\\

How test will be performed: Call Specular.get$\_$mirrored$\_$paths(wall, start, end)) and print the output as well as the expected results.

\end{enumerate}

\newpage
\begin{landscape}
\subsection{Traceability Between Test Cases and Modules}
\begin{table}[h!]
	\centering
	\begin{tabular}{|c|c|c|c|c|c|c|c|c|}
		\hline
		& \nameref{Point} & \nameref{wall} & \nameref{LP} & \nameref{intersection} & \nameref{ll} & \nameref{specular} & \nameref{Ainput} & \nameref{Aoutput} \\
		\hline
		Control    			& & & & & & &X&X\\ \hline
		Input Parameters   	& & & & & & &X& \\ \hline
		Point     			&X& & & & & & & \\ \hline
		Wall			 	& &X& & & & & & \\ \hline
		Floor Map	     	& & & & & & & &X\\ \hline
		Equation Finder	    & &X& & & & & & \\ \hline
		Linear Signal Path	& & &X& & & & & \\ \hline
		Intersection		& & & &X& & & & \\ \hline
		Linear Path Loss	& & & & &X& & & \\ \hline
		Specular Reflection & & & & & &X& & \\ \hline
		Line-Of-Sight Signal& & & & & & & &X\\ \hline
		First-Order Reflection Signal & & & & & & & &X\\ \hline
		Received Signal Strength & & & & & & & &X\\ \hline
		Output 				& & & & & & & &X\\ \hline
		Specification Parameters & & & & & & & &X\\ \hline
	\end{tabular}
	\caption{Traceability Between Test Cases and Requirements}
	\label{Table:A_trace}
\end{table}
\end{landscape}

\bibliographystyle{plainnat}

\bibliography{../../refs/References}

\newpage

\end{document}