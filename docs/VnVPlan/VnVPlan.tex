\documentclass[12pt, titlepage]{article}

\usepackage{booktabs}
\usepackage{longtable}
\usepackage{tabularx}
\usepackage{hyperref}
\hypersetup{
    colorlinks,
    citecolor=blue,
    filecolor=black,
    linkcolor=red,
    urlcolor=blue
}
\usepackage[round]{natbib}

%% Comments

\usepackage{color}

\newif\ifcomments\commentstrue

\ifcomments
\newcommand{\authornote}[3]{\textcolor{#1}{[#3 ---#2]}}
\newcommand{\todo}[1]{\textcolor{red}{[TODO: #1]}}
\else
\newcommand{\authornote}[3]{}
\newcommand{\todo}[1]{}
\fi

\newcommand{\wss}[1]{\authornote{blue}{SS}{#1}} 
\newcommand{\wxz}[1]{\authornote{cyan}{XZ}{#1}} 
\newcommand{\plt}[1]{\authornote{magenta}{TPLT}{#1}} %For explanation of the template
\newcommand{\an}[1]{\authornote{cyan}{Author}{#1}}

%% Common Parts

\newcommand{\progname}{ProgName} % PUT YOUR PROGRAM NAME HERE %Every program
                                % should have a name


\begin{document}

\title{Project Title: System Verification and Validation Plan for Radio Signal Strength Calculator} 
\author{Xingzhi Liu}
\date{\today}
	
\maketitle

\pagenumbering{roman}

\section{Revision History}

\begin{tabularx}{\textwidth}{p{3cm}p{2cm}X}
\toprule {\bf Date} & {\bf Version} & {\bf Notes}\\
\midrule
Oct 30, 2020 & 1.0 & First Draft of VnV Plan\\
\bottomrule
\end{tabularx}

\newpage

\tableofcontents

\listoftables

\listoffigures

\newpage

\section{Symbols, Abbreviations and Acronyms}

\renewcommand{\arraystretch}{1.2}
\begin{tabular}{l l} 
  \toprule		
  \textbf{symbol} & \textbf{description}\\
  \midrule 
  FR & Functional Requirement\\
  MG & Module Guide\\
  MIS & Module Interface Specification\\
  NFR & Non-Functional Requirement\\
  RSSC & Sadio SignalStrength Calculator\\
  SRS & Software Requirement Specification\\
  T & Test\\
  VV & Validation and Verification\\
  \bottomrule
\end{tabular}\\

\newpage

\pagenumbering{arabic}

This document records and presents the verification and validation plan for RSSC
to help ensure the program meets the requirements. General information including
a quick recall of RSSC's background is given in \autoref{GeneralInfo}. 
\autoref{plan} provides a plan for verification and \autoref{systd} describes 
the system tests, including tests for functional requirements and tests for non-
functional requirements.


\section{General Information}
\label{GeneralInfo}

\subsection{Summary}

The software to test is Radio Signal Strength Calculator (RSSC). The purpose of
RSSC is to simulate the radio signal propagation in the indoor environment defined
by the user, and find the analytical signal strength for the user. 

\subsection{Objectives}

The objective of this document is to build confidence in the software correctness.
To reach this objective, all functional and non-functional requirements will be
tested following the descriptions in this document. 

\subsection{Relevant Documentation}

\begin{itemize}
	\item 
	\href{https://github.com/XingzhiMac/CAS741-Proj/blob/master/docs/SRS/SRS.pdf}
	{SRS} 
\end{itemize}

\section{Plan}\label{plan}
	
\subsection{Verification and Validation Team}

\begin{itemize}
	\item Author: Xingzhi Liu
\end{itemize}

\begin{itemize}
	\item Primary Reviewer: Siddharth (Sid) Shinde
\end{itemize}

\begin{itemize}
	\item Reviewer: Leila Mousapour
\end{itemize}

\begin{itemize}
	\item Reviewer: Shayan Mousavi Masouleh
\end{itemize}

\begin{itemize}
	\item Dr. Spencer Smith
\end{itemize}

\subsection{SRS Verification Plan}

SRS will be done by team members reviewing the document. Team members can put
any comments, suggestions or questions in RSSC's Github repository as issues.
The author will respond to the issues and make modifications when needed.

\subsection{Design Verification Plan}

Design verification will be done by team members, by reviewing whether the steps 
of calculation in the software follows the physical model in SRS or not. 

\subsection{Implementation Verification Plan}

Implementation verification will be done by testing all the functional and non-
functional requirements. Descriptions of the tests can be found in \autoref{tfr} and \autoref{tnfr}. In addition, we will undergo static verification by checking
all the codes we build with Pylint. We will also conduct unit testing for modules
within the testing scope. Details for unit testing can be found in \autoref{unittest}.


\subsection{Automated Testing and Verification Tools}
\begin{itemize}
	\item Python Unittest
\end{itemize}
\begin{itemize}
	\item Pylint
\end{itemize}
\wss{The details of this section will likely evolve as you get closer to the
  implementation.}

\subsection{Software Validation Plan}

There are no plans for validation. We may be able to set up a simulation of radio
frequency signal on existing physics simulation tools and take the simulation 
result as a reference to evaluate RSSC's output, but different tools makes 
different assumptions to the indoor signal propagation model and we cannot 
change their assumptions. It is highly unlikely that we could find a tool 
that makes the same assumptions as RSSC. Therefore we do not have any reference 
to evaluate RSSC's correctness.


\section{System Test Description}
\label{systd}
	
\subsection{Tests for Functional Requirements}
\label{tfr}

Functional requirements for RSSC are given in 
\href{https://github.com/XingzhiMac/CAS741-Proj/blob/master/docs/SRS/SRS.pdf}
{SRS} section 5.1. There are 5 functional requirements for RSSC, from R1 to R5.
R1 and R2 are corresponding to inputs, while R3 to R5 are corresponding to outputs.
\autoref{area1} describes the input tests for R1 and R2; and \autoref{area2} 
describes the output tests for R3, R4 and R5.

\subsubsection{Area of Testing1 - Input}\label{Ainput}
\label{area1}

This test verifies the following requirements: \\
\indent  R1: RSSC takes input from the user;\\
\indent  R2: RSSC verifies user inputs.\\
		
\paragraph{Input Tests}

\begin{enumerate}

\item{Valid inputs\\}

Control: Manual
					
Initial State: Pending Input
					
Input:
$Pos_{tsm} = [0,0];$\\
$[Pos_{sp}] = [[10,0],[-10,0]];$\\
$[C] = [[1,1],[2,2],[1,3]];$\\
$[D] = [[2,2],[1,3],[1,1]];$\\
$[T] = [0.1,0.1,0.1];$\\
$[R] = [0.6,0.6,0.6];$\\
$P_{tsm}^{dBm} = 0;$\\
$f = 2.48\times10^{9};$\\

Output: Return an input success message in command line.

Test Case Derivation: RSSC correctly takes the inputs from the user
					
How test will be performed: Tester manually feeds the inputs into
RSSC and executes RSSC.
					
					
\item{Inconsistent input array sizes\\}

Control: Manual
					
Initial State: Pending Input
					
Input: 
$Pos_{tsm} = [0,0];$\\
$[Pos_{sp}] = [[10,0],[-10,0]];$\\
$[C] = [[1,1],[2,2],[1,3]];$\\
$[D] = [[2,2],[1,3]];$\\
$[T] = [0.1,0.1,0.1];$\\
$[R] = [0.6,0.6,0.6];$\\
$P_{tsm}^{dBm} = 0;$\\
$f = 2.48\times10^{9};$\\
					
Output: Return error of inconsistent array size in command line.

Test Case Derivation: Correct error message displays.

How test will be performed: Tester manually feeds the inputs into
RSSC and executes RSSC.


\item{Out-of-range position coordinates\\}

Control: Manual
					
Initial State: Pending Input
					
Input:
$Pos_{tsm} = [21,0];$\\
$[Pos_{sp}] = [[10,0],[-10,0]];$\\
$[C] = [[1,1],[2,2],[1,3]];$\\
$[D] = [[2,2],[1,3],[1,1]];$\\
$[T] = [0.1,0.1,0.1];$\\
$[R] = [0.6,0.6,0.6];$\\
$P_{tsm}^{dBm} = 0;$\\
$f = 2.48\times10^{9};$\\

Output: Return error of out-of-range position coordinates.

Test Case Derivation: Correct error message displays.
					
How test will be performed: Tester manually feeds the inputs into
RSSC and executes RSSC.


\item{Out-of-range transmitter power level\\}

Control: Manual
					
Initial State: Pending Input
					
Input:
$Pos_{tsm} = [0,0];$\\
$[Pos_{sp}] = [[10,0],[-10,0]];$\\
$[C] = [[1,1],[2,2],[1,3]];$\\
$[D] = [[2,2],[1,3],[1,1]];$\\
$[T] = [0.1,0.1,0.1];$\\
$[R] = [0.6,0.6,0.6];$\\
$P_{tsm}^{dBm} = 20;$\\
$f = 2.48\times10^{9};$\\

Output: Return error of out-of-range transmitter power level.

Test Case Derivation: Correct error message displays.
					
How test will be performed: Tester manually feeds the inputs into
RSSC and executes RSSC.


\item{Out-of-range signal frequency\\}

Control: Manual
					
Initial State: Pending Input
					
Input:
$Pos_{tsm} = [0,0];$\\
$[Pos_{sp}] = [[10,0],[-10,0]];$\\
$[C] = [[1,1],[2,2],[1,3]];$\\
$[D] = [[2,2],[1,3],[1,1]];$\\
$[T] = [0.1,0.1,0.1];$\\
$[R] = [0.6,0.6,0.6];$\\
$P_{tsm}^{dBm} = 20;$\\
$f = 2.48\times10^{12};$\\

Output: Return error of out-of-range signal frequency.

Test Case Derivation: Correct error message displays.
					
How test will be performed: Tester manually feeds the inputs into
RSSC and executes RSSC.

\wxz{TODO - learn how to write automatic testing scripts to generate random valid input
sets and feed into RSSC.}
\end{enumerate}

\subsubsection{Area of Testing2 - Output}\label{Aoutput}
\label{area2}

This test verifies the following requirements: \\
\indent  R3: RSSC shall find $P_{sp}^{dBm}$;\\
\indent  R4: RSSC shall verify $P_{sp}^{dBm}$;\\
\indent  R5: RSSC shall generate a file to store $P_{sp}^{dBm}$;\\

\paragraph{Output Test}

\begin{enumerate}

\item{Simple single valid output\\}

Control: Manual
					
Initial State: Pending Input
					
Input:
$Pos_{tsm} = [0,0];$\\
$[Pos_{sp}] = [[1,0]];$\\
$[C] = [ ];$\\
$[D] = [ ];$\\
$[T] = [ ];$\\
$[R] = [ ];$\\
$P_{tsm}^{dBm} = 0;$\\
$f = 2.48\times10^{9};$\\

Output: A .txt or .csv file with 1 line, showing the sampling point
position (1,0) and $P_{sp}^{dBm}$ that satisfies $P_{sp}^{dBm} 
\leq 0$. 

Test Case Derivation: RSSC returns the correct output.
					
How test will be performed: Tester manually feeds the inputs into
RSSC and executes RSSC.
					

\item{Simple multiple valid output\\}

Control: Manual
					
Initial State: Pending Input
					
Input:
$Pos_{tsm} = [0,0];$\\
$[Pos_{sp}] = [[1,0],[1,1]];$\\
$[C] = [ ];$\\
$[D] = [ ];$\\
$[T] = [ ];$\\
$[R] = [ ];$\\
$P_{tsm}^{dBm} = 0;$\\
$f = 2.48\times10^{9};$\\

Output: A .txt or .csv file with 2 lines. Line 1 shows the sampling point
position (1,0) and $P_{sp}^{dBm}$ that satisfies $P_{sp}^{dBm} 
\leq 0$. Line 2 shows the sampling point position (1,1) and $P_{sp}^{dBm}$ 
that satisfies $P_{sp}^{dBm} \leq 0$. 

Test Case Derivation: RSSC returns the correct output.
					
How test will be performed: Tester manually feeds the inputs into
RSSC and executes RSSC.
\wxz{TODO - learn how to write automatic testing scripts to generate random 
valid input sets and feed into RSSC.}

\end{enumerate}

\subsection{Tests for Non-Functional Requirements}
\label{tnfr}

Non-functional requirements are given in SRS section 5.2. There are 5 non-
functional requirements: Portable, Maintainable, and Understandable. The 
rest of this section provides detailed descriptions on how to test them.

\subsubsection{Portable}\label{Aport}

\paragraph{Portability Test} RSSC shall be able to run on different OS. 
We will test execute RSSC on both Windows and Linux Ubuntu.

\begin{enumerate}

\item{Portability on Windows 10\\}

Type: Manual
					
Initial State: Pending Input
					
Input: $Pos_{tsm} = [0,0];$\\
$[Pos_{sp}] = [[1,0]];$\\
$[C] = [ ];$\\
$[D] = [ ];$\\
$[T] = [ ];$\\
$[R] = [ ];$\\
$P_{tsm}^{dBm} = 0;$\\
$f = 2.48\times10^{9};$\\
					
Output: A .txt or .csv file with 1 line, showing the sampling point
position (1,0) and $P_{sp}^{dBm}$ that satisfies $P_{sp}^{dBm} 
\leq 0$. 
					
How test will be performed: Tester deploys RSSC on a machine with Windows 
10 OS and execute with the input set above. On success, RSSC should
provide the output as described.


\item{Portability on Linux Ubuntu\\}

Type: Manual
					
Initial State: Pending Input
					
Input: $Pos_{tsm} = [0,0];$\\
$[Pos_{sp}] = [[1,0]];$\\
$[C] = [ ];$\\
$[D] = [ ];$\\
$[T] = [ ];$\\
$[R] = [ ];$\\
$P_{tsm}^{dBm} = 0;$\\
$f = 2.48\times10^{9};$\\
					
Output: A .txt or .csv file with 1 line, showing the sampling point
position (1,0) and $P_{sp}^{dBm}$ that satisfies $P_{sp}^{dBm} 
\leq 0$. 
					
How test will be performed: Tester deploys RSSC on a virtual machine 
with Ubuntu 20.04 and execute with the input set above. On success, 
RSSC should provide the output as described.

\end{enumerate}

\subsubsection{Maintainable}\label{Amaintain}

\paragraph{Maintainability Test} Proper documents should be included
in this project. 

\begin{enumerate}
\item{Maintainability Test\\}

Type: Manual
					
Initial State: none
					
Input: none
					
Output: none
					
How test will be performed: Tester manually checks the contents in the
Github repo. On success, documents shall be uploaded following the schedule
of CAS741 and no issue shall be closed without a proper response.

\end{enumerate}

\subsubsection{Understandable}\label{Audst}

\paragraph{Understandability Test} Programs of RSSC should be organized,
well commented, and easy to understand.

\begin{enumerate}
\item{Understandability Test\\}

Type: Manual
					
Initial State: none
					
Input: none
					
Output: none
					
How test will be performed: Tester review the code and complete the survey
shown in \autoref{ut}.

\end{enumerate}

\begin{longtable}[h]{l l}
\label{ut}
	\begin{tabular}{l l} 
	\toprule		
	\textbf{Item} & \textbf{Score}\\
	\midrule 
	Variable names are rational and follow consistent conventions & \{0 - 5\}\\
	Functions are well commented on what they do & \{0 - 5\} \\
	Functions tasks are broken down / No super complicated functions & \{0 - 5\} \\
	No repeated chunks in the code & \{0 - 5\} \\
	Code is well organized and follows the order of instance models & \{0 - 5\} \\
	\bottomrule
	\caption{Understandability Survey} \label{Undgradesheet} \\
\end{tabular}\\
\end{longtable}

\subsection{Traceability Between Test Cases and Requirements}

\begin{table}[h!]
	\centering
	\begin{tabular}{|c|c|c|c|c|c|c|c|c|}
		\hline
		& R1 & R2 & R3 & R4 & R5 & Portable & Maintainable & Understandable\\
		\hline
		\ref{Ainput}    &X&X& & & & & & \\ \hline
		\ref{Aoutput}   & & &X&X&X& & & \\ \hline
		\ref{Aport}     & & & & & &X& & \\ \hline
		\ref{Amaintain} & & & & & & &X& \\ \hline
		\ref{Audst}     & & & & & & & &X\\ \hline
	\end{tabular}
	\caption{Traceability Between Test Cases and Requirements}
	\label{Table:A_trace}
\end{table}

\section{Unit Test Description}
\label{unittest}

\wss{Reference your MIS and explain your overall philosophy for test case
  selection.}  
\wss{This section should not be filled in until after the MIS has
  been completed.}

\subsection{Unit Testing Scope}

\wss{What modules are outside of the scope.  If there are modules that are
  developed by someone else, then you would say here if you aren't planning on
  verifying them.  There may also be modules that are part of your software, but
  have a lower priority for verification than others.  If this is the case,
  explain your rationale for the ranking of module importance.}

\subsection{Tests for Functional Requirements}

\wss{Most of the verification will be through automated unit testing.  If
  appropriate specific modules can be verified by a non-testing based
  technique.  That can also be documented in this section.}

\subsubsection{Module 1}

\wss{Include a blurb here to explain why the subsections below cover the module.
  References to the MIS would be good.  You will want tests from a black box
  perspective and from a white box perspective.  Explain to the reader how the
  tests were selected.}

\begin{enumerate}

\item{test-id1\\}

Type: \wss{Functional, Dynamic, Manual, Automatic, Static etc. Most will
  be automatic}
					
Initial State: 
					
Input: 
					
Output: \wss{The expected result for the given inputs}

Test Case Derivation: \wss{Justify the expected value given in the Output field}

How test will be performed: 
					
\item{test-id2\\}

Type: \wss{Functional, Dynamic, Manual, Automatic, Static etc. Most will
  be automatic}
					
Initial State: 
					
Input: 
					
Output: \wss{The expected result for the given inputs}

Test Case Derivation: \wss{Justify the expected value given in the Output field}

How test will be performed: 

\item{...\\}
    
\end{enumerate}

\subsubsection{Module 2}

...

\subsection{Tests for Nonfunctional Requirements}

\wss{If there is a module that needs to be independently assessed for
  performance, those test cases can go here.  In some projects, planning for
  nonfunctional tests of units will not be that relevant.}

\wss{These tests may involve collecting performance data from previously
  mentioned functional tests.}

\subsubsection{Module ?}
		
\begin{enumerate}

\item{test-id1\\}

Type: \wss{Functional, Dynamic, Manual, Automatic, Static etc. Most will
  be automatic}
					
Initial State: 
					
Input/Condition: 
					
Output/Result: 
					
How test will be performed: 
					
\item{test-id2\\}

Type: Functional, Dynamic, Manual, Static etc.
					
Initial State: 
					
Input: 
					
Output: 
					
How test will be performed: 

\end{enumerate}

\subsubsection{Module ?}

...

\subsection{Traceability Between Test Cases and Modules}

\wss{Provide evidence that all of the modules have been considered.}
				
\bibliographystyle{plainnat}

\bibliography{../../refs/References}

\newpage

\section{Appendix}

This is where you can place additional information.

\subsection{Symbolic Parameters}

The definition of the test cases will call for SYMBOLIC\_CONSTANTS.
Their values are defined in this section for easy maintenance.

\subsection{Usability Survey Questions?}

\wss{This is a section that would be appropriate for some projects.}

\end{document}